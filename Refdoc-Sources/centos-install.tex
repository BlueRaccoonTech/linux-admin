%----------------------------------------------------------------------------------------
%	Section 1.1 - CentOS Linux 8
%----------------------------------------------------------------------------------------

\subsubsection{Choosing CentOS Version}
Upon downloading CentOS, you are given an option to download CentOS Linux or CentOS Stream. While for personal projects I would be interested in using CentOS Stream to potentially gain access to newer sources, for the purposes of running and administering a company server, the only sensible option is CentOS Linux. CentOS Stream is a rolling distribution, designed to help contribute to future releases of Red Hat Enterprise Linux - and rolling distributions aren’t always the most stable, which means a higher likelihood of downtime that could have been avoided.
In regards to OS version, I will be rolling with CentOS 8. CentOS 8 would give us full updates until May 2024, and maintenance updates until May 31, 2029. On the other hand, we would no longer receive full updates for CentOS 7 starting Q4 of 2020, and would no longer receive maintenance updates starting June 30th, 2024. 
While the rollout for CentOS 8 has been claimed to not exactly be the best, I feel like this additional challenge shouldn’t cause too much of a problem.

\subsubsection{Booting into Anaconda}
The CentOS 8 installer is known as Anaconda, which starts up with the install media. The first screen is relatively straightforward - as long as you speak English and live in the United States, it’s probably best to stick with the defaults here. You are then taken to a page that may seem a tad overwhelming at first – but fear not! Take it step by step and it’s relatively simple.

\subsubsection{Anaconda - Localization}
The Localization column is the most straightforward - it defaults to English (US) for the keyboard and Americas/New York for the timezone. If either of these need to be changed (whether you live in another timezone or you use a different keyboard layout), these can be changed by clicking on them and making the appropriate changes.

\subsubsection{Anaconda - System}
Under the System column, select “Network \& Host Name.” Set your desired system name under Host Name. I chose buizel.server.acme for this server. If you have a DHCP server running on the network, you can go ahead and enable the Ethernet at this point. (If you’ve ever connected a device to the Internet and didn’t need to give it an IP address, subnet, gateway, and DNS server addresses, you have a DHCP server on your network.) Otherwise, select “Configure…” and enter the appropriate information under IPv4 settings (and IPv6, if your network has it enabled.) Below is an example static IP setup with IPv4.

Back at the Summary screen, select “Installation Destination”. You can leave everything at this screen at the defaults, but if you know what you’re doing, you can also select “Custom” and make whatever adjustments you see fit to the partitioning of the disk. This, however, is beyond the scope of this server install - the defaults will be fine in our case.

\subsubsection{Anaconda - Software}
We’ll need to make a change under the Software column. You don’t want to install a GUI with the server - it’s an unnecessary waste of resources and you’ll find yourself wondering why you did it. Click “Software Selection” and select “Server” for the Base Environment. Don’t worry about checking any of the boxes under “Additional software” - we can install these later.

\subsubsection{Anaconda - Begin Installation}
At this point, the installer will allow you to continue to installing the system - press “Begin Installation” to kickstart the installation process. At this point, the installer will begin to install files to disk and you will be given the option to create a root password and create a user. Do both of these - make sure the root password is extremely strong, as this password would give someone complete and total control over the server. Under User Creation, make sure “Make this user administrator” is checked, and set just as strong of a password, as this account would be able to log into the root account with its credentials.
\subsubsection{Finishing the Installation}
At this point, you can pretty much just sit back as the install completes! You’ll know it’s finished when it says “CentOS Linux is now successfully installed and ready for you to use! Go ahead and reboot to start using it!” Press the blue Reboot button to restart into the new OS.
Once you reach this screen, you’ll want to login.
Then, once logged in, you’ll want to update anything that might need updating using the command \verb|sudo yum update|.